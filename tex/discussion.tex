\renewcommand{\thesubsection}{\Alph{subsection}}
\setcounter{subsection}{1} 
\subsection{Dynamic Length Estimation}
To improve the accuracy of length estimation, we expanded Eq. (\ref{eq:model}) by adding the terms $p^2$ and $d^2$ and increasing the parameters as follows:
\begin{equation}
\label{eq:model_2d(1)}
F = (b_5p^2 + b_4pd + b_3d^2 + b_2p+b_1d+b_0)d
\end{equation}
As a result, with respect to the measurements by the linear encoder,the dynamic length estimation was achieved with maximum errors of 1.12 $\%$ for PAM-A, 0.773 $\%$ for PAM-B, 1.01 $\%$ for PAM-B, and 0.755 $\%$ for PAM-D respectively, and with root mean squared errors of 0.633 $\%$ for PAM-A, 0.353 $\%$ for PAM-B, 0.548 $\%$ for PAM-C, and 0.435 $\%$ for PAM-D respectively. The errors were reduced as expected for all PAMs. 

We also tried another approach by introducing a cubic polynomial model and increasing the parameters as follows:
\begin{equation}
    \label{eq:model_3d}
    F = (c_4p^3+c_3p^2d+c_2pd^2+c_1d^2+c_0)d
\end{equation}
As predicted, the root mean squared errors decreased to 0.516 $\%$ for PAM-A, 0.484 $\%$ for PAM-B, 0.500 $\%$ for PAM-C, and 0.606 $\%$ for PAM-D respectively. 
However, even though the maximum errors decreased to 1.22 $\%$ for PAM-A and 0.951 $\%$ for PAM-C respectively, they actually increased to 1.41 $\%$ for PAM-B and 1.99 $\%$ for PAM-D respectively. This result suggests that, even if the coefficients of the model equation are determined experimentally, the degrees must be carefully determined based on previous studies to accurately express intrinsic characteristics of the PAM. For example, the newly added term $p^3$ may have amplified the error of the pressure sensor. When applying our model to a reflex mechanism, it will also be necessary to carefully consider the contribution of each term to the accuracy of the length estimation based on the reliability of the force and pressure sensors used.

Wickramatunge et al. proposed separating the parameters $a_i$ into contraction ones $a^c_i$ and extension ones $a^e_i$ to reflect the hysteresis of the PAM\cite{spring}. They also suggested using different parameters for low-pressure and high-pressure ranges to further improve the accuracy. However, our model did not adopt these suggestions and simplifies the length estimation method by using the same parameters across the entire pressure range, regardless of contraction or expansion. This is because our model is supposed to be applied to the reflex mechanism. If the parameters have to be switched depending on the situation, it would be difficult for the reflex mechanism to respond quickly to disturbances. Musculoskeletal robots often carry microcomputers on their structures, so the employed length estimation method should be simple for efficient operation given the limited computational resources.



\subsection{Reaching task}
In the reaching task, the errors of the estimation by the model greatly increased compared to those in the dynamic length estimation in the previous section. This is because, as the PAM contracts, the fishing line loses contact with the shaft, resulting in slackness and failure to send voltage signals. In Fig.\ref{fig:reaching_error}, the rate of change in the length estimation by the model decreases at around 2.7 second because the fishing line loses contact with the shaft and the deformation is estimated as zero from this point onward. In fact, for the agonist muscle, the maximum error after the fishing line slackened at 2.7 second grew significantly to 8.82 $\%$ with the root mean squared error of 5.94 $\%$, whereas prior to 2.7 second, the errors were smaller, with the maximum error of 2.99 $\%$, and the root mean squared error  of 1.71 $\%$. Compared to the fiber sensor, the errors were still large even before 2.7 second. One possible cause lies in the process of converting the strain gauge voltage to force. The voltage values fluctuated significantly due to slight positional shifts of the fishing line, and the error in the slope of voltage to force $q$ might have been amplified during the estimation procedure.

The looseness of the strain gauge poses a disadvantage  when it comes to continuously tracking the total length of the PAM, but it contributes to creating more biologically inspired behavior. In the human body, a muscle spindle is aligned parallel to a muscle and monitors change in its length, but when the muscle contracts, the muscle spindle becomes unloaded, leading to the cessation of neural discharging activity\cite{spindle}. One of the ultimate goals of our research is to propose a possible operating principle of the reflex mechanism in the human body by incorporating it into a musculoskeletal robot. Our length estimation method is designed in pursuit of this purpose, so it is desirable to develop a local control system that only responds when the muscle is suddenly stretched just as the human neural system does. In this sense, the error during muscle contraction does not need to be a primary concern.

\subsection{Stretch Reflex}
In the reflex experiment, since the fiber sensor directly measure the diameter change of the PAM, it could accurately track the length change and capture the impact as a spike in length velocity. With our model, however, the fluctuations in the pressure sensor and the strain gauge were transmitted to the estimated length and thus the length velocity was overestimated, so the impact was not separated clearly in the velocity field. This made it difficult to set the threshold and the feedback gain, thereby increasing the possibility of mistriggering the reflex or causing the reflex to continue triggering itself. This issue might be mitigated by applying a low-pass filter to the estimated velocity.

In addition to the Ia reflex pathway that detects overstretch of a muscle with a muscle spindle, the human muscle has the Ib reflex pathway that perceives excessive tension with a Golgi tendon organ. Therefore, pressure sensors, tension sensors and length sensors are required to realize the coexistence of the two pathways in a musculoskeletal robot driven PAMs. The advantage of our approach is that the length sensors can be removed by allocating its function to the pressure and tension sensors.