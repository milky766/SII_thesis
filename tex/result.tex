\subsection{Parameter Identification}
Fig. \ref{fig:length_pressure} shows the relationship between the pressure $p$ and the natural length $l_n$ of the PAMs. As assumed, there is a tendency for $l_n$ to decrease linearly with $p$ within the range of the tested pressure. The dashed lines in Fig. \ref{fig:length_pressure} represent

\begin{figure}[t]
    \hfill
    \begin{minipage}{\columnwidth}
        \centering
        \includegraphics[width=\columnwidth]{fig/length_pressure.pdf} 
        \caption{Relationship between Pressure and Natural Length}
        \label{fig:length_pressure}
        \vspace{1em} 
        \includegraphics[width=\columnwidth]{fig/20231124_5_4s_2d_ieeesensors1.pdf}
        \caption{Relationship between Force and Deformation at Each Pressure (PAM-B)}
        \label{fig:pam_b_static1}
    \end{minipage}
    \hspace{0.05\textwidth} 
\end{figure}

\noindent the fitted lines using the least squares method, expressed by Eq. (\ref{eq:L_n}).

Fig. \ref{fig:pam_b_static1} shows the result of the static loading experiment for PAM-B. The red and blue points represent the data during expansion and contraction respectively, and the green dashed lines represent the solutions $d$ to Eq. (\ref{eq:model}), which is given by substituting the acquired parameters $a_i$ and the measured $f$ and $p$.
Generally, PAMs exhibit hysteresis due to friction, so the data differ between expansion and contraction processes. We only present the static loading experimental result for PAM-B because of the space constraint, but similar results were obtained for the other PAMs.


% \begin{textblock*}{\columnwidth}(11cm,3cm) 
% \begin{figure}
%     \centering
%     \includegraphics[width=\columnwidth]{fig/length_pressure.pdf}
%     \caption{Relationship between Pressure and Natural Length}
%     \label{fig:length_pressure}
% \end{figure}
% \end{textblock*}


% \begin{textblock*}{\columnwidth}(11cm,12cm) 
% \begin{figure}
%    \centering
%    \includegraphics[width=\columnwidth]{fig/20231124_5_4s_2d_ieeesensors1.pdf}
%    \caption{Relationship between Force and Deformation at Each Pressure (PAM-B)}
%    \label{fig:pam_b_static1}
% \end{figure}
% \end{textblock*}

% % 画像にかからないようにテキスト位置を調整
% \vspace{40cm} % 必要に応じて調整してください

% \begin{figure}[H]
%     \centering
%     \includegraphics[width=\columnwidth]{fig/length_pressure.pdf}
%     \caption{Relationship between Pressure and Natural Length}
%     \label{fig:length_pressure}
% \end{figure}

% \begin{figure}[H]
%    \centering
%    \includegraphics[width=\columnwidth]{fig/20231124_5_4s_2d_ieeesensors1.pdf}
%    \caption{Relationship between Force and Deformation at Each Pressure (PAM-B)}
%    \label{fig:pam_b_static1}
% \end{figure}


\begin{figure*}[h]
   \begin{center}
       \begin{minipage}[t]{\columnwidth} 
           \centering
           \includegraphics[keepaspectratio, width=\columnwidth]{fig/20231207_1_4s_by_5_2d_ieeesensors1.pdf}
           \caption{Dynamic Length Estimation (PAM-B, Rubber)}
           \label{fig:pam_b_dynamic}
       \end{minipage}
       \hfill
       \begin{minipage}[t]{\columnwidth} 
           \centering
           \includegraphics[keepaspectratio, width=\columnwidth]{fig/20231220_2_s_by_2_2d_ieeesensors1.pdf}
           \caption{Dynamic Length Estimation (PAM-D, Silicon)}
           \label{fig:pam_d_dynamic}
       \end{minipage}
   \end{center}
\end{figure*}

\vspace{1cm}

% \begin{figure}[H]
%    \centering
%    \includegraphics[width=\columnwidth]{fig/20231207_1_4s_by_5_2d_ieeesensors1.pdf}
%    \caption{Dynamic length estimation (PAM-B, Rubber)}
%    \label{fig:pam_b_dynamic}
% \end{figure}
\subsection{Error Evaluation} 
Fig. \ref{fig:pam_b_dynamic} and Fig. \ref{fig:pam_d_dynamic} show the dynamic length estimation result for PAM-B and PAM-D, respectively.
With the proposed method, with respect to the measurements by the linear encoder, the length estimation was achieved with maximum errors of $1.72\%$ for PAM-A, $1.19\%$ for PAM-B, $1.18\%$ for PAM-B, and  $1.65\%$ for PAM-D respectively , and with mean squared errors of $0.861\%$ for PAM-A, $0.653\%$ for PAM-B, $0.683\%$ for PAM-C, and $0.846\%$ for PAM-D respectively.

\subsection{Reaching Task}
Table. \ref{tab:PAM_reflex} shows the parameters for the length estimation of the agonist and antagonist muscles. The considerable difference in the voltage-force slope $q$  between the two PAMs is attributable to the varying sensitivities of the handmade force gauges.

\begin{table}[H]
    \centering
    \caption{Parameters for Length Estimation} 
    \resizebox{\columnwidth}{!}{%
    \begin{tabular}{c|ccccccc}
        \hline
        PAM & $m$ & $h$ & $a_3$ & $a_2$ & $a_1$ & $a_0$ & $q$ \\
        \hline \hline
        Agonist & -60.1 & 170.1 & -0.201 & 7.00 & 0.256 & 0.911 & $2.25 \times 10^{-3}$\\
        Antagonist & -70.3 & 178.5 & 0.871 & 1.24 & -0.129 & 22.4 & $4.33 \times 10^{-3}$ \\
        \hline
    \end{tabular}
    } 
    \label{tab:PAM_reflex}
\end{table}


\subsection{Stretch Reflex}
Table \ref{tab:reflex_para} shows the velocity threshold $V_{thr}$, which were determined by assessing the magnitude of the impact in advance. 

\begin{table}[h]
    \centering
    \caption{Parameters for Stretch Reflex} 
    \begin{tabular}{c|cc}
        \hline
        PAM &$V_{thr} [\si{mm/s}]$&$ k [\si{GPa\cdot s}]$\\
        \hline \hline
        Model & 70 & 1/800\\
        Fiber Sensor & 25 & 1/300\\
        \hline
    \end{tabular}
\label{tab:reflex_para}
\end{table}
% \begin{figure}[H]
%    \centering
%    \includegraphics[width=\columnwidth]{fig/20231220_2_s_by_2_2d_ieeesensors1.pdf}
%    \caption{Dynamic length estimation (PAM-D, Silicon)}
%    \label{fig:pam_d_dynamic}
% \end{figure}

% \clearpage






% \begin{table}[h]
%     \centering
%     \caption{Maximum error and root mean squared \\percentage error of dynamic length estimation }
%     \resizebox{\columnwidth}{!}{
%         \begin{tabular}{c|cccccc}
%             \hline
%             PAM & \begin{tabular}[c]{@{}c@{}}Maximum Error[$\%$]\\(Eq.(\ref{eq:model})) \end{tabular} & \begin{tabular}[c]{@{}c@{}}Root Mean\\Squared Error[$\%$]\\(Eq.(\ref{eq:model})) \end{tabular} & \begin{tabular}[c]{@{}c@{}}Maximum Error[$\%$]\\(Eq.(\ref{eq:model_2d1})) \end{tabular} & \begin{tabular}[c]{@{}c@{}}Root Mean\\Squared Error[$\%$]\\(Eq.(\ref{eq:model_2d1})) \end{tabular}&\begin{tabular}[c]{@{}c@{}}Maximum Error[$\%$]\\(Eq.(\ref{eq:model_3d})) \end{tabular}& \begin{tabular}[c]{@{}c@{}}Root Mean\\Squared Error[$\%$]\\(Eq.(\ref{eq:model_3d}))\\
%             \hline \hline
%             A & 1.72&0.861&1.12&0.633& &&\\
%             B & 1.19&0.653 &0.773&0.353&&&\\
%             C & 1.18&0.683&1.01&0.548& &&\\
%             D & 1.65 & 0.846 &0.755& 0.435& &&\\
%             \hline
%             \hline
%         \end{tabular}
%     }
%     \label{tab:error}
% \end{table}

% \begin{table}[h]
%     \centering
%     \caption{Maximum error and root mean squared \\percentage error of dynamic length estimation }
%     \resizebox{\columnwidth}{!}{
%         \begin{tabular}{c|ccccccc}
%             \hline
%             PAM & \begin{tabular}[c]{@{}c@{}}Maximum Error[$\%$]\\(Eq.(\ref{eq:model})) \end{tabular} & \begin{tabular}[c]{@{}c@{}}Root Mean\\Squared Error[$\%$]\\(Eq.(\ref{eq:model})) \end{tabular} & \begin{tabular}[c]{@{}c@{}}Maximum Error[$\%$]\\(Eq.(\ref{eq:model_2d(1)})) \end{tabular} & \begin{tabular}[c]{@{}c@{}}Root Mean\\Squared Error[$\%$]\\(Eq.(\ref{eq:model_2d(1)})) \end{tabular}&\begin{tabular}[c]{@{}c@{}}Maximum Error[$\%$]\\(Eq.(\ref{eq:model_3d})) \end{tabular}&\begin{tabular}[c]{@{}c@{}}Root Mean\\Squared Error[$\%$]\\(Eq.(\ref{eq:model_3d})) \end{tabular} \\
%             \hline \hline
%             A & 1.72&0.861&1.12&0.633&1.22&0.516&\\
%             B & 1.19&0.653 &0.773&0.353&1.41&0.484&\\
%             C & 1.18&0.683&1.01&0.548&0.951&0.500&\\
%             D & 1.65 & 0.846 &0.755& 0.435&1.99&0.606&\\
%             \hline
%         \end{tabular}
%     }
%     \label{tab:error}
% \end{table}


% \subsection{推定パラメータ同定}
% 図\ref{fig:length_pressure}は,4種の空気圧人工筋の圧力$P$と自然長$L_n$の関係である.
% 仮定通り,圧力$P$に対して自然長$L_n$が線形的に減少する傾向が見られる.
% 図中の点線は,最小二乗法により各データ郡にフィッティングした(\ref{eq:L_n})式である.

% 一方,図\ref{fig:pam_b_static1}はPAM-Bに対する推定パラメータ同定実験の結果である.
% ただし,赤色の点が膨張時のデータ,青色の点が収縮時のデータ,緑色の点線が取得した推定パラメータ$a_i$による推定値を表す.
% 一般に,空気圧人工筋は摩擦によるヒステリシスを有するので,膨張時と収縮時でデータが異なる.
% % また,図\ref{fig:pam_b_static2}は図\ref{fig:pam_b_static1}中の5つの圧力に対する各データをまとめて三次元空間上に示した図である.
% 本論文では,紙幅上PAM-Bに対する実験結果のみ示すが,他の空気圧人工筋に対しても同様の結果が得られた.

% \begin{figure}[H]
%    \centering
%    \includegraphics[width=\columnwidth]{fig/length_pressure.pdf}
%    \caption{Relationship between pressure and unstretched length}
%    \label{fig:length_pressure}
% \end{figure}

% \begin{figure}[H]
%    \centering
%    \includegraphics[width=\columnwidth]{fig/20231124_5_4s_se_2d_g.pdf}
%    \caption{Relationship between force and stretched length at each pressure (PAM-B)}
%    \label{fig:pam_b_static1}
% \end{figure}

% % \begin{figure}[H]
% %    \centering
% %    \includegraphics[width=\columnwidth]{fig/20231124_5_4s_se_3d_g.pdf}
% %    \caption{Relationship between pressure, force and stretched length (PAM-B)}
% %    \label{fig:pam_b_static2}
% % \end{figure}


% \subsection{動的測定実験} 
% 図\ref{fig:pam_b_dynamic}は,,PAM-Bに対して取得した推定パラメータを用い,動的に長さを推定した結果である.
% 提案手法では,表\ref{tab:error}の第2列および第3列に示す最大誤差および平均平方二乗誤差率で,それぞれ動的に長さを推定できた.


% \begin{figure}[H]
%    \centering
%    \includegraphics[width=\columnwidth]{fig/20231207_1_4s_by_5_2d_g.pdf}
%    \caption{Result of dynamic length estimation (PAM-B)}
%    \label{fig:pam_b_dynamic}
% \end{figure}

% \begin{table}[h]
%     \centering
%     \caption{Maximum error and root mean squared \\percentage error of dynamic length estimation }
%     \resizebox{\columnwidth}{!}{
%         \begin{tabular}{c|ccccc}
%             \hline
%             \begin{tabular}[c]{@{}c@{}}空気圧\\人工筋\end{tabular} & \begin{tabular}[c]{@{}c@{}}最大誤差[$\%$]\\(二次式) \end{tabular} & \begin{tabular}[c]{@{}c@{}}平均平方\\二乗誤差率[$\%$]\\(二次式) \end{tabular} & \begin{tabular}[c]{@{}c@{}}最大誤差[$\%$]\\(三次式) \end{tabular} & \begin{tabular}[c]{@{}c@{}}平均平方\\二乗誤差率[$\%$]\\(三次式) \end{tabular} \\
%             \hline \hline
%             PAM-A & 0.975&0.510&1.22&0.516& \\
%             PAM-B & 0.955&0.510 &1.41&0.484&\\
%             PAM-C & 1.07&0.580&0.951&0.500& \\
%             PAM-D & 1.41 & 0.475 &1.99& 0.606& \\
%             \hline
%         \end{tabular}
%     }
%     \label{tab:error}
% \end{table}
