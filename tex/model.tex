Regarding a PAM as a spring\cite{spring}, its length is expressed as the sum of natural length and deformation
\begin{equation}
   l = l_n + d
\label{eq:estimation}
\end{equation}
where $l$ is PAM length, $l_n$ is natural length, defined as length with no external force at pressure $p$, and $d$ is deformation from natural length, respectively.
 
Assuming that $l_n$ is a linear function of $p$ within a certain pressure range, it can be given as
\begin{equation}
l_n = mp + h
\label{eq:L_n}
\end{equation}
where $m$ and $h$ can be determined by static tensile tests.
 
Introducing the PAM's nonlinearity into the spring and assuming that the spring constant is the function of $p$ and $d$, the force $f$ is given by
\begin{equation}
\label{eq:model}
f = (a_3pd + a_2p + a_1d + a_0)d
\end{equation}
The terms $pd^2$ and $d^2$ can be found in Chou et al.'s fundamental model\cite{chouMeasurementModelingMcKibben1996}, which capture the essential dynamic properties of the PAM. The term $pd$  comes from Tondu et al.'s model\cite{ModelingControl} to more accurately reflect differences in the shape of the PAM. The term $d$ is added based on Ferraresi et al.'s model\cite{Comparison} to account for differences in material. The constants $a_0 \sim a_3$ in Eq. (\ref{eq:model}) are determined by the static tensile tests, and based on them, $d$  can be calculated from the measured $p$ and $f$ by solving Eq. (\ref{eq:model}).

To examine the effect of the dimension and the number of parameters in the model equation, quadratic Eq.(\ref{eq:model_2d(1)}) and cubic Eq. (\ref{eq:model_3d}) were also adapted, and the estimation errors were compared with those of the original model equation.
\begin{equation}
   \label{eq:model_2d(1)}
   f = (b_5p^2 + b_4pd + b_3d^2 + b_2p+b_1d+b_0)d
   \end{equation}

   \begin{equation}
      \label{eq:model_3d}
      f = (c_4p^3+c_3p^2d+c_2pd^2+c_1d^2+c_0)d
  \end{equation}