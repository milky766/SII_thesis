Soft robots are expected to achieve adaptability to environments like living organisms\cite{rus_design_2015}. 
They have already been developed to coexist with humans \cite{Compac,polygerinos_soft_2015}, and have been used to understand biological intelligence through a constructive methodology\cite{hosoda,marchese_autonomous}.
As actuators in musculoskeletal robots, pneumatic artificial muscles (PAMs) are often used\cite{mirvakili_artificial}.
PAMs have several advantages over conventional actuators, such as a superior power-to-mass ratio\cite{Dynamic}, high compliance\cite{SDCharacteristics}, and low cost and ease of production\cite{ashwin_survey_2018}.
% However, their inherent nonlinearity places heavy computational demands on a central control system.
% To overcome this hurdle, it is proposed to integrate reflex mechanisms found in living organisms into musculoskeletal robots\cite{takahashi}.
However, PAMs have nonlinearity because they consist of elastic materials, which places heavy computational demands on a central control system.

To overcome this hurdle, it is proposed to integrate reflex mechanisms found in living organisms into musculoskeletal robots \cite{takahashi}.
Reflex mechanisms enable local control systems to swiftly respond to environmental changes without commands from a central control system, thereby reducing its computational load. 

One of the reflex mechanisms, a stretch reflex, detects the sudden change of muscle length \cite{kandel}, but directly measuring PAM length with a sensor poses several challenges\cite{nakajima}.
First, since a reflex action occurs instantaneously, it might lead to significant problems such as slackness in a wire encoder or light screen in a laser sensor, 
which could prevent accurate measurement of the PAM's length. 
Second, because a length sensor needs to be posed at both PAM's ends, it could limit robot design. 
Third, sensor stiffness could impair the flexibility of the PAM.

This paper presents a method to estimate the PAM length from its pressure and force instead of directly measuring it with the goal of incorporating the stretch reflex into musculoskeletal robots.
The model views a PAM as a nonlinear spring, and the spring constant is dependent on its deformation and pressure. 
We determined the dimensions of the spring constant based on prior models and determined its coefficients experimentally.
The effectiveness of the model was demonstrated by evaluating the errors in length estimation when pressure varied sinusoidally.
The model was subsequently incorporated to the stretch reflex mechanism and required to maintain a robot arm in a fixed position while the falling mass generated an impact.
The dynamic response of the reflex with the model was then compared to that with a conventional sensor. 
Theoretical models of the PAM often have difficulty in accurately capturing individual differences in properties because they rely on almost immeasurable parameters, such as the total length of braided fibers or their braiding angle\cite{motion}. However, our approach overcomes this limitation by experimentally determining coefficients for each PAM and thus reflecting their characteristic. Applying our model not only offers broad applicability to PAMs of various materials and shapes, but also allows the sensors to gather at one end of a PAM, simplifying the robotic design. 